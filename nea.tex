\documentclass{article}

\usepackage{hyperref}
\usepackage{biblatex}
\addbibresource{nea.bib}

\author{Nils André-Chang}
\title{NEA: OCR Exam Reference Language as a Language, an implementation}

\begin{document}

{\huge THIS COPY OF THE DOCUMENT IS A DRAFT AND IS NOT MEANT TO BE MARKED}

\maketitle

\tableofcontents

Disclaimer: In this document, the author refers to themselves as `we' however,
there is only a singular author, and the use of this pronoun is not meant as an
indication of there being multiple authors.

An implementation of the OCR Exam Reference Language as defined by
\textcite{j277, h446}.

\section{Analysis}

\subsection{The problem and its computational solution}

Every year in the UK, thousands of students take computer science as a subject
for their secondary school education; in the summer of 2019, 11124 took it for
A Levels, 3098 for AS Levels, and 80027 for GCSEs
\cite{jcqalevel19, jcqgcse19}.

In order to assess the ability of the students in the subject, 2 methods are
used: examination and non exam assessment (NEA) in the form a programming
project. To assess students during exams, exam boards produce questions that
use and ask the student to use a pseudocode that is language agnostic and has
been predefined in the specification of the exam board.  This is done for
clarity and consistency among other reasons \cite{h446, j276, j277}.

Such an exam board is OCR: in \textcite{h446, j276, j277} a pseudocode is
defined, in \textcite{j277}, it was renamed to ``OCR Exam Reference Language''.

The problem with having a relatively strictly defined language is that although
it is called a "pseudocode" (at least in older specifications) and it is
mentioned that ``Learners [...] may provide answers in any style of pseudocode
they choose providing its meaning could be reasonably inferred by a competent
programmer.'', it is often that students lose mark dude to the style of
pseudocode they chose not being understood by the examiner for various reasons.
As a result of this, it is preferable for students to write their answers in a
style that is as close as possible to the "pseudocode" defined by the exam
board to avoid losing marks.

From here on out, what has been described as the `pseudocode' will now be
referred to as a language for 2 reasons: firstly, We believe that as an
inherent result of the language used in OCR exams being formally defined, it is
no longer a pseudocode but in fact a programming language. Secondly, in the
newest computer science specification, \textcite{j277}, what was previously
described as a pseudocode, is now known as the ``OCR Exam Reference Language''
(We suspect it is because of the realisation of point 1 by OCR although there
is no evidence to this).

Unfortunately, learning a language is complicated and one of the best ways to
learn one is to practice with it. However because the language is only defined
in the specifications of exam boards, there is no way to practice using it, as
there is no way to execute any code written using the syntax of this language.

Another problem that stems from the use of this language as described is that
there is no way to check in a reliable way if code written using it is correct.
Whether that is when a student is practicing for exams or when writing exam
papers themselves. In the June 2018 A Level Paper 2, there was a mistake in the
exam paper students had to take \cite{ocrpec18}.

We believe that a solution for the problems described would be to implement the
language, as described in the specifications and as used in exams, such that
users can run code that was written using the language.

If there was the possibility to run the code, it will allow students to use the
language when going through the GCSE and A Level courses which will allow them
to learn the language well and then be able to use it confidently within the
exam. This will improve their ability and allow them to score higher in exams.
Being able to execute code written in this language will also allow to check
answers when practising giving students practical ways to verify what they are
doing; this is especially important at GCSE where students are not necessarily
confident enough to be able to tell for themselves if something is correct and
why it is correct. Lastly being able to execute code written in this language
will also allow to check code that was written using it for example for the
purpose of exam papers removing the possibility of mistakes creeping up like
they did in the June 2018 A Level paper.

Another advantage of having a working implementation of this language is that
students will no longer need to learn 2 languages: a programming language and
the OCR Exam Reference Language. They will be able to only learn the OCR Exam
Reference Language because they will now be able to use it as a proper
programming language and write programs using it.

Although this project will focus on the OCR Exam Reference Language, this
section in particular applies to other exam boards. The problems described here
are not specific to OCR.

\subsection{Stakeholders}



\subsection{Existing solutions}

\subsection{Essential Features}

% TODO implement other exam boards as a bonus

\subsection{Limitations}

Performance

\subsection{Hardware and Software requirements}

The language needs to be widely available (as most programming languages are)
and as a result needs to have the least amounts of requirements as possible.
Due to the implementation of the language being in the form of an interpreter
and there being no low level constructs defined (within the OCR spec), the
language doesn't require to run without an operating system (This would be a
requirement if firmwares, operating system and other low level software was
written using the language). As a result a standard operating system will be
required.

The operating systems that will be primarily supported are Linux, MacOS and
Windows (in that order) however the interpreter will be written in a cross
platform fashion (by using libraries to abstract platform specific code) such
that it is most likely going to work on any other modern operating system such
as the BSDs.

If a playground is offered, a browser implementing the standard browser
specifications should be required and no specific operating system will be
required. The web version could be implemented using web assembly in which case
a modern browser supporting WebAssembly will be required. If the web version is
implemented by running the interpreter on a server, the server will require one
of the operating system aforementioned.

There are no hardware requirements other than dependencies on already mentioned
software requirements. Interpreters are not resource intensive programs and can
even execute on most low power micro controllers\cite{micropython}.

\subsection{Success Criteria}

\printbibliography[heading=bibintoc]

\end{document}
