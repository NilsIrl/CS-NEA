\documentclass{article}

\usepackage{hyperref}
\usepackage{biblatex}
\addbibresource{nea.bib}

\usepackage{adjustbox}

\author{Nils André-Chang}
\title{NEA: OCR Exam Reference Language as a Language, an implementation}

\begin{document}

{\huge THIS COPY OF THE DOCUMENT IS A DRAFT AND IS NOT MEANT TO BE MARKED}

\maketitle

\tableofcontents

In this document, the author refers to themselves as `we' however, there is
only a singular author, and the use of this pronoun is not meant as an
indication of there being multiple authors.

An implementation of the OCR Exam Reference Language as defined by
\textcite{j277, h446}.

\section{Analysis}

\subsection{The problem and its computational solution}

Every year in the UK, thousands of students take computer science as a subject
for their secondary school education; in the summer of 2019, 11124 took it for
A Levels, 3098 for AS Levels, and 80027 for GCSEs
\cite{jcqalevel19, jcqgcse19}.

In order to assess the ability of the students in the subject, 2 methods are
used: examination and non exam assessment (NEA) in the form a programming
project. To assess students during exams, exam boards produce questions that
use and ask the student to use a pseudocode that is language agnostic and has
been predefined in the specification of the exam board.  This is done for
clarity and consistency among other reasons \cite{h446, j276, j277}.

Such an exam board is OCR: in \textcite{h446, j276, j277} a pseudocode is
defined, in \textcite{j277}, it was renamed to ``OCR Exam Reference Language''.

The problem with having a relatively strictly defined language is that although
it is called a ``pseudocode" (at least in older specifications) and it is
mentioned that ``Learners [...] may provide answers in any style of pseudocode
they choose providing its meaning could be reasonably inferred by a competent
programmer.'', it is often that students lose mark dude to the style of
pseudocode they chose not being understood by the examiner for various reasons.
As a result of this, it is preferable for students to write their answers in a
style that is as close as possible to the ``pseudocode" defined by the exam
board to avoid losing marks.

From here on out, what has been described as the `pseudocode' will now be
referred to as a language for 2 reasons: firstly, We believe that as an
inherent result of the language used in OCR exams being formally defined, it is
no longer a pseudocode but in fact a programming language. Secondly, in the
newest computer science specification, \textcite{j277}, what was previously
described as a pseudocode, is now known as the ``OCR Exam Reference Language''
(We suspect it is because of the realisation of point 1 by OCR although there
is no evidence to this).

Unfortunately, learning a language is complicated and one of the best ways to
learn one is to practice with it. However because the language is only defined
in the specifications of exam boards, there is no way to practice using it, as
there is no way to execute any code written using the syntax of this language.

Another problem that stems from the use of this language as described is that
there is no way to check in a reliable way if code written using it is correct.
Whether that is when a student is practicing for exams or when writing exam
papers themselves. In the June 2018 A Level Paper 2, there was a mistake in the
exam paper students had to take \cite{ocrpec18}.

% TODO how is computation better than other solutions.

We believe that a solution for the problems described would be to implement the
language, as described in the specifications and as used in exams, such that
users can run code that was written using the language.

If there was the possibility to run the code, it will allow students to use the
language when going through the GCSE and A Level courses which will allow them
to learn the language well and then be able to use it confidently within the
exam. This will improve their ability and allow them to score higher in exams.
Being able to execute code written in this language will also allow to check
answers when practising giving students practical ways to verify what they are
doing; this is especially important at GCSE where students are not necessarily
confident enough to be able to tell for themselves if something is correct and
why it is correct. Lastly being able to execute code written in this language
will also allow to check code that was written using it for example for the
purpose of exam papers removing the possibility of mistakes creeping up like
they did in the June 2018 A Level paper.

Another advantage of having a working implementation of this language is that
students will no longer need to learn 2 languages: a programming language and
the OCR Exam Reference Language. They will be able to only learn the OCR Exam
Reference Language because they will now be able to use it as a proper
programming language and write programs using it.

Although this project will focus on the OCR Exam Reference Language, this
section in particular applies to other exam boards. The problems described here
are not specific to OCR.

% TODO reference to the sections

From here on out, the result of this project will be referred to as an
implementation of the language, this is because the specifics of the
implementation have not been well defined yet and will be in the features and
the design section.

\subsection{Stakeholders}

One of the main stakeholders for the project are the users. There will be
many different types of users but a majority of them will fall under one of
four different categories: exam board and resource producers, student, and
teacher.

The exam boards and resource producers will make a similar use of the
implementation of the language: they will use it to create resources, one
mainly in the form of exams and the other will form a wide variety of
resources: exam style questions, practice activities and other learning
resources. They can use the implementation to verify code they show the
students is correct. \Textcite{pattis88} found that only 20\% of textbooks
implemented binary search correctly and the 2018 A Level OCR paper contained an
error in the code they presented students suggesting errors in code presented
to students is common and solutions need to be found. We believe that using
an implementation of the language in one such solution and will reduce the
occurrence of this mistake. Exam boards have previously been fined for mistakes
in exams \cite{ofqual20180702} and resource producers could also be sued if the
content they produce lead to students failing their qualifications. For this
reason, using this an implementation of the language will not only improve the
accuracy or resources it will also reduce costs (in forms of fines) for exam
boards and resource producers.

Another category user of the implementation of the language will be students
who are studying computer science in secondary school. They will be able to use
the implementation to check their answers to questions as well as to write
programs as part of their learning. We believe that this will help boost their
grades in 2 ways: firstly, being able to check their answer will allow them go
to a lot more quickly through questions and learn the fundamentals of
programming a lot more quickly and be more confident with their learning as
they can prove that what they have learned works; secondly, having to learn
only a single language will reduce the amount of learning necessary which will
allow them to have a more focused learning and know the content they need to
know better.

Lastly, teachers like students will be able to use the implementation of the
language to mark their students and make sure the answers are correct and will
be able to use it to plan their lessons and demo some of the programs to their
students.

Other than users, stakeholders may include hosting companies especially if a
server side ``playground'' is developed as part of the project. However,
regardless of if and how services provided as part of this project are hosted,
the hosting provider will probably be automated and the existence of this
project will be negligible to them.

% None user stakeholders (possible hosting provider)

\subsection{Existing solutions}

Currently what is done is that when people write code in the OCR Exam Reference
Language is that they can only check it manually, this can be considered to be
a solution to making mistakes however it is error prone and doesn't provide
much.

There are however other software that have been written in the past that are
much more similar to what this project aims to offer.

\subsubsection{Pseudo Code Interpreter}

\Textcite{jacobsieradzki18} is an iPad app that allows to execute pseudo code
for OCR exams.

However, it has many disadvantages that I hope this project will address.
Starting with the syntax, it uses a syntax that although is inspired from the
OCR pseudo code guide is not like the OCR pseudocode guide and as a result
cannot be used to learn the language properly nor can it be used to check for
mistake. Another feature of \textcite{jacobsieradzki18} that makes it
unsuitable to solve the problem demonstrated above is that it is only available
as an iPad app and as a result it cannot be used across a wide variety of
devices and for many different purposes; this ties in with the fact the
language is not implemented like other languages: it is not possible to use it
standalone. For example, when input is asked from the user a special interface
is shown which is not suitable for use as a programming language.

I will be able to use \textcite{jacobsieradzki18} to inspire a possible user
interface for the implementation.

\subsubsection{Pseudocompiler}

\Textcite{pseudocompiler} ``is a (pretty) spec-compliant implementation of
OCR's provisional "pseudocode" specification''.

It seems to have the features that this project aims for including a playground
which is a website to test out a language. However, we were not able to use it
and the link to the playground is invalid. Additionally, although it discusses
the possible use of LLVM as a target, for the moment it only targets JavaScript
which means that it requires a JavaScript runtime to execute making it less
cross platform.

Considering there are no instructions on how to use it and no demos or examples
of its uses whatsoever on its homepage, it was not possible for us to look at
its features and as a result we do not have much to go off of.

\subsubsection{Pseudocode-Compiler}

\Textcite{pseudocode-compiler} is a compiler ``that compiles IGCSE pseudocode
to LLVM IR''.

This project compiles to LLVM IR which allows it to support many platforms, as
many as LLVM supports, however it compiles IGCSE pseudocode which is not
suitable for our use case.

\subsubsection{Pseudocode-Transpiler}

\Textcite{pseudocode-transpiler} is a compiler that ``compiles pseudocode into
python''. It uses regular expressions to evaluate statements.

This project compiles IGCSE pseudocode using regular expressions which is not a
reliable method, additionally, it compiles to python which is itself an
interpreted language (at least its main implementation is), which means that
running code using this method will be slow and probably inaccurate.

To conclude there are many different approaches to writing an implementation of
any given pseudocode. It is possible to compile to different ``targets'' such
as the JVM, python, LLVM or JavaScript; each have their advantage and
disadvantage, some being more cross-platform than others, more efficient than
others.

It is also possible to interpret the language and there are different types of
user interfaces.

\paragraph{Other projects:}

\begin{itemize}
    \item{https://github.com/Sherlemious/IGCSE-CS-PC-Transpiler}
\end{itemize}

\subsection{Essential Features}

% TODO implement other exam boards as a bonus

In of itself, the project is quite self contained and there aren't many
features apart from the core itself: implement a language.

It is possible however to separate the core of the language into different
section; for example, the GCSE features and the A Level (Object-oriented
programming) features can be separated.

Other features which we are unlikely to be able to implement but may be done as
a bonus in order to turn this implementation into a more usable language is
tooling. Here is a list of tooling we could implement although it is quite
unlikely this stage is reached.

\begin{itemize}
    \item{Syntax highlighting for already existing editors}
    \item{A playground, an platform to test a language}
    \item{An Language Server}
\end{itemize}

\subsection{Limitations}

% TODO: students will probably still need to learn another language alongside

A limitation of this solution which we propose is that although this language
will be usable as a general purpose language, students will likely benefit from
also learning another language that is more widely used and is supported by
more libraries. This means that using this implementation in order to avoid
learning 2 languages is not necessarily beneficial.

Another limitation of this solution is that the OCR Exam Reference Language is
a language designed to be as neutral as possible and has as its only intended
purpose to represent simple algorithms for exams. As a result, the language is
very limited and cannot be used for much without extensions that add features
to it and it is therefore not necessarily beneficial to learn it as a language.

One of the most important features of a language is its community: it
allows to have easily available libraries to make development faster without
having to rewrite code that has already been written before. By using a niche
language like this one, a developer is giving up on all the libraries that can
make their life easier.

% TODO: reference the design section here

Lastly, the tooling. Another important part of the user experience of a
language is the tooling that comes with that language making it easier to use.
Although we hope to have syntax highlighting through tree-sitter (see Design
section) and possibly an LSP, the tooling will be greatly reduced compared to
other languages which means that the development experience will be less
enjoyable and more difficult leading to users not being able to write as high
quality code.

\subsection{Hardware and Software requirements}

The language needs to be widely available (as most programming languages are)
and as a result needs to have the least amounts of requirements as possible.
Due to the implementation of the language being in the form of an interpreter
and there being no low level constructs defined (within the OCR spec), the
language doesn't require to run without an operating system (This would be a
requirement if firmwares, operating system and other low level software was
written using the language). As a result a standard operating system will be
required.

The operating systems that will be primarily supported are Linux, MacOS and
Windows (in that order) however the interpreter will be written in a cross
platform fashion (by using libraries to abstract platform specific code) such
that it is most likely going to work on any other modern operating system such
as the BSDs.

If a playground is offered, a browser implementing the standard browser
specifications should be required and no specific operating system will be
required. The web version could be implemented using web assembly in which case
a modern browser supporting WebAssembly will be required. If the web version is
implemented by running the interpreter on a server, the server will require one
of the operating system aforementioned.

There are no hardware requirements other than dependencies on already mentioned
software requirements. Interpreters are not resource intensive programs and can
even execute on most low power micro controllers\cite{micropython}.

\subsection{Success Criteria}

In order to test the software and validate the success criteria in an objective
way, a testing script \texttt{neasuccess} will be written before the
implementation has started to be written. This test script will run the
implementation with different inputs and verity that the output is the one
expected.

The success test script will be modular and be able to support many different
tests. It will work in the following way: In the \texttt{tests} directory,
there will be files with the extension \texttt{.input} and \texttt{.output}.
The success test script will run the interpreter with as input the content of
the \texttt{.input} files and check if the output is the same as the file with
the same basename but as extension \texttt{.output} file.

An argument can be given to the success test script to only use the tests with
the basename given.

The following success criteria are derived from the pseudocode guide in
\textcite{h446}. They are ordered such that later tests can include features
from previous tests. Due to the nature of a programming language, it is very
possible that some of these features will all be implemented at once,
especially considering some features are really built-in functions more than
anything else and as such may not require to be core language features.

\begin{table}
    \begin{adjustbox}{center}
        \begin{tabular}{|l|l|}
            \hline
            Criteria & How to evidence \\
            \hline
            Being able to read a file & Run the interpreter with a file as input \\
            \hline
            Working sequential instructions & Run \texttt{./neasuccess seq} \\
            \hline
            Comments & Run \texttt{./neasuccess comments} \\
            \hline
            Outputting to screen & Run \texttt{./neasuccess print} \\
            \hline
            Variables & Run \texttt{./neasuccess variables} \\
            \hline
            Iteration --- Count Controlled & Run \texttt{./neasuccess forloop}
            \\
            \hline
            Selection & Run \texttt{./neasuccess selection} \\
            \hline
            Logical Operators & Run \texttt{./neasuccess logic} \\
            \hline
            Iteration --- Condition Controlled & Run \texttt{./neasuccess
            whileloop} \\
            \hline
            String Handling & Run \texttt{./neasuccess strings} \\
            \hline
            Subroutines & Run \texttt{./neasuccess subroutines} \\
            \hline
            Subroutines --- Passing values by value or by reference & Run
            \texttt{./neasuccess byvalue\_byref} \\
            \hline
            Arrays & Run \texttt{./neasuccess arrays} \\
            \hline
            Reading from files & Run \texttt{./neasuccess readfile} \\
            \hline
            Writing to files & Run \texttt{./neasuccess writefile} \\
            \hline
            Objects --- Attributes & Run \texttt{./neasuccess attributes} \\
            \hline
            Objects --- Methods & Run \texttt{./neasuccess methods} \\
            \hline
            Object --- Access level & Run \texttt{./neasuccess access} \\
            \hline
            Object --- Constructors & Run \texttt{./neasuccess constructors} \\
            \hline
            Object --- Inheritence & Run \texttt{./neasuccess inheritance} \\
            \hline
        \end{tabular}
    \end{adjustbox}
    \caption{Success Criteria}
\end{table}

\section{Design}

There are 2 main ways to be able to write language: as an interpreter or as a
compiler. Additionally there are also different flavours of JIT compilers.

We have decided to opt for an interpreter due to the language's dynamic nature
as well as because interpreters are less complex.

The task of interpreting source code can be separated into 2 major parts:

\begin{enumerate}
    \item{Parsing---Converting source code into an abstract syntax tree (AST)}
    \item{Interpreting---Interpreting the abstract syntax tree}
\end{enumerate}

At first we were hoping to use tree-sitter as our parsing system because it
would give us syntax highlighting inside of editors without having to write a
regex based grammar or another parser. However writing tree-sitter grammars is
not an easy task and is described as having ``a difficult learning
curve''\cite{ts_creating_parsers}.

This is why we have decided to use a different method for parsing the code. We
settled on using parser combinators with the help of the parser combinator
framework \textcite{nom}. This library was chosen because it is the most widely used
parsing framework for rust (our implementation language) and the author has
experience having used it while going through \textcite{eopl}.

The implementation language will be rust because it is a language the author is
familiar with and because it is a high performance language as it compiles to
machine code. This is very important when writing an interpreter because it
will allow to have acceptable performance. Writing an interpreter in an
interpreted language would accumulate overhead and result in bad performance.

\subsection{Language Grammar}

\noindent
Program ::= \{Statement\}\textsuperscript{*}

\noindent
Statement ::= \texttt{global} Identifier = Expression\\
Statement ::= Identifier \texttt{=} Expression\\
Statement ::= \texttt{array} Identifier\texttt{[}\{Expression\}\textsuperscript{+(\texttt{,})}\texttt{]}\\
Statement ::= Expression\\
Statement ::= \texttt{for} Identifier \texttt{=} Expression \texttt{to} Expression List-of-Statements \texttt{next} Identifier\\
Statement ::= \texttt{while} Expression List-of-Statements \texttt{endwhile}\\
Statement ::= \texttt{do} List-of-Statements \texttt{until} Expression\\

\noindent
Statement ::= \texttt{if} Expression \texttt{then} List-of-Statements \{\texttt{elseif} Expression \texttt{then} List-of-Statements\}\textsuperscript{*} \{\texttt{else} List-of-Statements\}\textsuperscript{?} \texttt{endif}\\
Statement ::= \texttt{switch} Expression\texttt{:} \{\texttt{case} Expression\texttt{:} List-of-Statements\}\textsuperscript{*} \{\texttt{default:} List-of-Statements\}\textsuperscript{?} \texttt{endswitch}\\

\noindent
Statement ::= \texttt{function} Identifer(\{Identifier\{:byVal|:byRef\}\}\textsuperscript{*(,)}) List-of-Statements endfunction\\
Statement ::= \texttt{procedure} Identifer(\{Identifier\{:byVal|:byRef\}\}\textsuperscript{*(,)}) List-of-Statements endfunction\\
Statement ::= \texttt{return} Expression\\

\noindent
List-of-Statements ::= ()\\
List-of-Statements ::= (Statement . List-of-Statements)\\

\noindent
Expression ::= Constant\\
Expression ::= Identifer(\{Expression\}\textsuperscript{*(,)})\\
Expression ::= Identifier.Field\\
Expression ::= Identifier.Method(\{Expression\}\textsuperscript{*(,)})\\

\noindent
Expression ::= Expression AND Expression\\
Expression ::= Expression OR Expression\\
Expression ::= NOT Expression\\

\noindent
Expression ::= Expression == Expression\\
Expression ::= Expression != Expression\\
Expression ::= Expression \textless{} Expression\\
Expression ::= Expression \textless= Expression\\
Expression ::= Expression \textgreater{} Expression\\
Expression ::= Expression \textgreater= Expression\\

\noindent
Expression ::= Expression + Expression\\
Expression ::= Expression - Expression\\
Expression ::= Expression * Expression\\
Expression ::= Expression / Expression\\
Expression ::= Expression MOD Expression\\
Expression ::= Expression DIV Expression\\
Expression ::= Expression \textasciicircum{} Expression\\

Note on string literals: the unicode characters `“` and `”` seem to be used to
delimit string literals. Is that normal??

In the examples given, different unicode characters are used, that's a bit
wacky.

Escaping doesn't seem to exist either

We'll assume identifiers cannot start with numbers (see alter, maybe this can
be changed)

What do we consider whitespace? is a zero width space whitespace?

Do we want to implement closures?? (for language? for implementation? for parsing? for interpreting?)

What are valid identifiers? We'll follow C with: only: letters, digits and
underscore, but first character must be letter or underscore.

Test cases shouldn't be called '*.input' they can contain actual programs so
that doesn't make sense.

%We do not have operator precedence (help)

%Rust doesn't support backtracking in macros:
%https://github.com/rust-lang/rust/issues/24827, https://github.com/rust-lang/rust/issues/42838
so we had to inverse the direction of our ops macro

%Use pretty_assertions to speed up AST testing

Fuzz the parser (idk why, we should just do it cause that sounds cool)

Switch statement implemented as a bunch of if else

This language makes no sense. For example, switch statements have a colon after
the expression. WHY?

On (physical) page 39, there is a 1, after `case "B"`, wtf

Switch statement if given complex expression will compute expression at every
check

Log:

%Changed from ../ to CARGO_MANIFEST_DIR

\printbibliography[heading=bibintoc]

\end{document}
