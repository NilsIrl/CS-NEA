
%%%%%%%%%%%%%%%%%%%%%%%%%%%%%%%%%%%%%%%%%%%%%%%%%%%%%%%%%%%%%%%%%%%%%%%%

In order to test the parser we are going to use the built-in unit testing
feature of Rust because they allow to simplify the process of testing with all
the already existing tools provided by rust, and is less error prone than
rolling out our own testing solution, using for example a custom script.

For example, we will be able to view our code coverage, a measure of how much
of the code is being tested by tests. Using this we will be able to add tests
to handle the under tested cases and have a measure of how much of the code is
tested and hence most likely to be correct.

By being able to quantitatively measure our testing strategy we can assure
ourself of the accuracy and reliability of our program.
